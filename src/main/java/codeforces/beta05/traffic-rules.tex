\setuplayout
	[backspace=20mm,
	 width=middle,
	 topspace=15mm,
	 height=fit,
	 header=0cm,
	 headerdistance=0cm,
	 footer=\lineheight,
	 footerdistance=4mm]
	 
\setupheader[text][state=high]
\setupfootertexts[][pagenumber][pagenumber][]
\setupwhitespace[line]
\setupmathematics[integral=limits]
\let\le\leqslant
\let\ge\geqslant
\starttext
\starttitle[title={Traffic Rules}]

For convenience let's rename the inputs. Let the maximum car speed be
$V\unit{km/h}$, the maximum speed allowed at the sign be $W\unit{km/h}$ and the
overall road length $L\unit{km}$.

Let $v(t)$ be the car speed at the time $t$ and

\startformula\ell(t) = \int_0^t v(\tau) \dd\tau\stopformula

be the distance that the car has covered at that moment. Out task is to organize
car movement so that the time $t_*$ such that $\ell(t_*) = L$ is minimal.

There are following restrictions on the car movement.

\startitemize[n]
\item The starting speed is zero: $v(0) = 0$.
\item The maximum car speed cannot surpass $V$: $v(t) \le V$.
\item The speed at the sign doesn't surpass $W$, that is, when $\ell(t) = d$
the speed $v(t) \le W$. Let $t_d$ be the time of reaching the sign.
\item The car can't accelerate and decelerate faster than $a\unit{km/h2}$:
$|v'(t)| \le a$.
\stopitemize

The solution splits into a variety of cases which we group based on whether we
decelerate at the sign and whether we reach the maximum speed.

\startsection[title={We don't decelerate at the sign}]

First we'll consider the solution itself for this case and then when it occurs.

Since we don't decelerate at the sign in this case, the optimal car movement is
as follows. We accelerate with the maximum acceleration $a$ until the maximum
speed $V$ is reached and then proceed with the speed $V$ until the road ends.

\startformula v(t) = \startmathcases
\NC at, \NC 0\le t\le t_V,\NR
\NC V, \NC t > t_V.\NR
\stopmathcases \stopformula

It might happen that the road ends before we reach the maximum speed though.

Given the optimal car movement, the time to reach the maximum speed

\startformula t_V = \frac{V}{a}\stopformula

and the distance that the car covers during this time is

\startformula L_V = \frac{V^2}{2a}. \stopformula

If $L < L_V$, that is when \inframed{$2aL < V^2$}, then the road ends before we
reach the full speed and

\startformula L = \ell(t_*) = \frac{at_*^2}{2},\quad\text{thus}\quad
\mcframed{t_* = \sqrt{\frac{2L}{a}}}.\stopformula

\stopsubsection

In the opposite case \inframed{$2aL \ge V^2$} we have

\startformula L = \ell(t_*) = \int_0^{t_V} a\tau\dd\tau + \int_{t_V}^{t_*}
V\dd\tau = \frac{at_V^2}{2} + V(t_* - t_V). \stopformula

Here we express

\startformula t_* = \frac{1}{V}\cdot\left(L - \frac{at_V^2}{2}\right)
 + t_V = \mcframed{\frac{L}{V} + \frac{V}{2a}}. \stopformula

When does the car not decelerate at the sign? First, if $W \ge V$, clearly we
don't have to decelerate. Otherwise we would cover distance

\startformula L_W = \frac{W^2}{2a} \stopformula

in order to reach speed $W$ and if $d \le L_W$, that is, $2ad\le W^2$, we would
not accelerate enough to violate the speed restriction at the sign. To
summarize, we don't decelerate at the sign when

\startformula \mcframed{W\ge V\quad\text{or}\quad 2ad \le W^2}.\stopformula

\stopsection

\startsection[title={We have to decelerate at the sign}]

In this more complicated case the optimal car movement looks as follows. First
we accelerate to speed $V$, drive with that speed, then decelerate to speed $W$
at the sign. After the sign we again accelerate to speed $V$ and drive until the
road ends. Again, we might not hit speed $V$ before the sign as well as after
the sign.

Let's first find the time $t_d$ to reach the sign.

\startsubsection[title={Maximum speed reached before the sign}]

If we hit the maximum speed before reaching the sign, the optimal speed

\startformula v(t) = \startmathcases
\NC at, \NC 0\le t\le t_V,\NR
\NC V, \NC t_V < t \le t_s,\NR
\NC W - a(t-t_d), \NC t_s < t \le t_d,\NR
\stopmathcases\stopformula

where $t_s$ is the time we start decelerating.

We have already found the time $t_V$ and the distance $L_V$ of accelerating to
speed $V$.

\startformula t_V = \frac{V}{a},\quad L_V = \frac{V^2}{2a}.\stopformula

Since we expect continuity of $v(t)$ in the point $t_s$, we have

\startformula W - a(t_s - t_d) = V,\quad\text{and the deceleration time}\quad
\quad t_d - t_s = \frac{V-W}{a}.\stopformula

The deceleration distance is

\startformula L_{VW} = \int_{t_s}^{t_d} v(\tau)\dd\tau = \frac{V^2 - W^2}{2a}.
\stopformula

From this we get the following expression for the distance to the sign.

\startformula d = \ell(t_d) = L_V + (t_s - t_V) \cdot V + L_{VW} = \frac{2V^2 -
W^2}{2a} + (t_d - t_V - (t_d - t_s))\cdot V = \frac{2V^2 - W^2}{2a} + \left(t_d
- \frac{2V - W}{a}\right). \stopformula

Therefore the time to reach the sign in this case is

\startformula \mcframed{t_d = \frac{2V - W}{a} + \frac{1}{V}\cdot \left(d -
\frac{2V^2 - W^2}{2a}\right)}. \stopformula

The formula is effective when the speed $V$ is reached, that is, when $L_V +
L_{VW} \le d$, or

\startformula \mcframed{2V^2 - W^2 \le 2ad}. \stopformula

\stopsubsection

\startsubsection[title={Maximum speed not reached before the sign}]

If we don't hit the maximum speed before reaching the sign, the optimal speed

\startformula v(t) = \startmathcases
\NC at, \NC 0\le t\le t_X, \NR
\NC W - a(t-t_d), \NC t_X < t \le t_d,\NR
\stopmathcases \stopformula

where $t_X$ is the moment when the acceleration stops and we start decelerating
before the sign. Let's get the expression for the distance to the sign.

\startformula d = \ell(t_d) = \int_0^{t_X} a\tau\dd\tau + \int_{t_X}^{t_d} (W -
a(\tau - t_d))\dd\tau. \stopformula

The first integral equals $\frac{at_X^2}{2}$. We change the variable in the
second integral: $u = W - a(\tau - t_d)$, $\dd\tau = -\frac{\dd u}{a}$.

\startformula \int_{t_X}^{t_d} (W - a(\tau - t_d))\dd\tau = \int_{W - a(t_X -
t_d)}^W u\cdot \left(-\frac{\dd u}{a}\right). \stopformula

The point $t_X$ is defined to satisfy equality $W - a(t_X - t_d) = at_X$, which
we'll use in the integration limit.

\startformula \int_{t_X}^{t_d} (W - a(\tau - t_d))\dd\tau = -\frac{1}{a} \cdot
\int_{at_X}^W u\dd u = \frac{1}{a} \cdot \int_W^{at_X} u\dd u = \frac{(a^2 t_X^2
- W^2)}{2a} = \frac{at_X^2}{2} - \frac{W^2}{2a}.\stopformula

In total for the $\ell(t_d)$ we have

\startformula d = \ell(t_d) = at_X^2 - \frac{W^2}{2a}. \stopformula

From this we can express the time $t_X$.

\startformula t_X = \sqrt{\frac{d}{a} + \frac{W^2}{2a^2}} = \frac{\sqrt{4ad +
2W^2}}{2a}. \stopformula

From the condition on $t_X$ we can find already $t_d$.

\startformula t_d = 2\left(t_X - \frac{W}{2a}\right) = \mcframed{\frac{\sqrt{4ad
+ 2W^2} - W}{a}}. \stopformula

From now on we consider the time $t_d$ to reach the sign a known quantity. We
can concentrate on finding the rest of the time needed to reach to destination.

\stopsubsection

\startsubsection[title={Maximum speed reached after the sign}]

If the car reaches the maximum speed after the sign before finishing the ride,
the speed function looks as follows.

\startformula v(t) = \startmathcases
\NC W + a(t-t_d), \NC t_d < t \le t_{WV},\NR
\NC V, \NC t_{WV} < t \le t_*,\NR
\stopmathcases \stopformula

where $t_{WV}$ is the time the car reaches the maximum speed $V$.

The time to accelerate from the speed $W$ to $V$ and the distance covered during
the acceleration is clearly the same as the time and distance for deceleration.

\startformula t_{WV} - t_d = \frac{V-W}{a},\quad L_{WV} = \frac{V^2 - W^2}{2a}.
\stopformula

Since the overall distance from the sign to the end is $L-d$, we have the
following condition for the case: $L_{WV} \le L - d$, that is, $\inmframed{V^2 -
W^2 \le 2a(L - d)}$.

We find $t_*$ in this case by expanding the definition $\ell(t_*) = L$.

\startformula L_{WV} + V(t_* - t_{WV}) + d = L, \stopformula

that is,

\startformula \mcframed{t_* = t_d + \frac{V-W}{a} + \frac{1}{V}\cdot \left(L - d
- \frac{V^2 - W^2}{2a}\right)}. \stopformula

\stopsubsection

\startsubsection[title={Maximum speed not reached after the sign}]

In this case the optimal movement speed is

\startformula v(t) = W + a(t-t_d)\quad\text{for } t_d < t \le t_*. \stopformula
From this we get the following expression for the distance covered after the
sign:

\startformula \int_{t_d}^{t_*} (W + a(\tau - t_d))\dd\tau. \stopformula

We know that this distance equals $L-d$ though. Let's change the integration
variable with substitution $u = W + a(\tau - t_d)$, $\dd\tau = \frac{1}{a}\dd
u$.

\startformula L-d = \int_W^{W - a(t_* - t_d)} \frac{u\dd u}{a} =
\frac{1}{2a}\cdot \left((W + a(t_* - t_d))^2 - W^2\right). \stopformula

From this we get the formula for $t_*$ in the final case:

\startformula \mcframed{t_* = t_d + \frac{\sqrt{W^2 + 2a(L-d)} - W}{a}}.
\stopformula

\stopsubsection \stopsection \stoptitle \stoptext